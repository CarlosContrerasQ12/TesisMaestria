% !TeX spellcheck = en_US
\documentclass[11pt]{report}
\usepackage[utf8]{inputenc}
\usepackage{amsmath}
\usepackage{amsthm}
\usepackage{amssymb}
\usepackage{graphicx}
\graphicspath{ {images/} }
\usepackage{caption}
\usepackage{tocloft}
\usepackage{parskip}
\usepackage{subcaption}
\usepackage[shortlabels]{enumitem}
\usepackage{float}
\usepackage{color}   %May be necessary if you want to color links
\usepackage{hyperref}
\usepackage{comment}
\usepackage{xcolor}
\usepackage{soul}
\usepackage[toc,titletoc,title]{appendix}

\makeatletter
\newcommand{\hlLarge}[1]{%
	\setbox\@tempboxa\hbox{#1}%
	\ifdim\wd\@tempboxa>\linewidth
	\noindent
	\colorbox{yellow}{%
		\parbox{\dimexpr\linewidth-2\fboxsep}{#1}%
	}%
	\else
	\noindent
	\colorbox{yellow}{#1}%
	\fi}%Highlighter.
\makeatother
\usepackage{framed}
\colorlet{shadecolor}{yellow}

\DeclareRobustCommand{\hlgreen}[1]{{\sethlcolor{green}\hl{#1}}}
\newcommand{\hlc}[2][]{\hl{#2}(\hlgreen{#1})}

\hypersetup{
	colorlinks=true, %set true if you want colored links
	linktoc=all,     %set to all if you want both sections and subsections linked
	linkcolor=green,  %choose some color if you want links to stand out
	bookmarksopen=true,
}
\hypersetup{linktocpage}
\usepackage[open,openlevel=1]{bookmark}

\usepackage[Lenny]{fncychap}


\usepackage[width=150mm,top=35mm,bottom=25mm,bindingoffset=6mm]{geometry}
\usepackage{fancyhdr}
\pagestyle{fancyplain}% <- use fancyplain instead fancy
\fancyhf{}
\fancyhead[R]{\thepage}
\renewcommand{\headrulewidth}{0pt}
\setlength{\headheight}{14pt}

\newtheorem{theorem}{Theorem}[section]
\newtheorem{definition}{Definition}[section]
\newtheorem{corollary}{Corollary}[theorem]
\newtheorem{lemma}[theorem]{Lemma}
\newtheorem{assumptions}[definition]{Assumptions}

\usepackage{bm}
\newcommand{\bbR}{\mathbb{R}}
\newcommand{\bfX}{\mathbf{X}}
\newcommand{\bfAlp}{\bm{\alpha}}
\renewcommand\qedsymbol{$\blacksquare$}
\newcommand{\dpartial}[2]{\frac{\partial #1}{\partial #2}}
\newcommand{\norm}[1]{\left\lVert#1\right\rVert}
\DeclareMathOperator{\Tr}{Tr}
\DeclareMathOperator*{\argmin}{arg\,min}

\setlength{\parindent}{0cm}

\usepackage{mathtools}

\NewDocumentCommand{\expect}{ e{^} s o >{\SplitArgument{1}{||}}m }{%
	\operatorname{\mathbb{E}}%     the expectation operator
	\IfValueT{#1}{{\!}^{#1}}% the measure of the expectation
	\IfBooleanTF{#2}{% *-variant
		\expectarg*{\expectvar#4}%
	}{% no *-variant
		\IfNoValueTF{#3}{% no optional argument
			\expectarg{\expectvar#4}%
		}{% optional argument
			\expectarg[#3]{\expectvar#4}%
		}%
	}%
}
\NewDocumentCommand{\expectvar}{mm}{%
	#1\IfValueT{#2}{\nonscript\;\delimsize\vert\nonscript\;#2}%
}
\DeclarePairedDelimiterX{\expectarg}[1]{[}{]}{#1}


\usepackage[sorting=none]{biblatex}
%\addbibresource{references.bib}
\addbibresource{Probabilidad.bib}
\addbibresource{PDEs ML.bib}
\addbibresource{ControlOptimizacion.bib}
\addbibresource{BSDEs.bib}
\addbibresource{MFG.bib}

\title{Deep learning method for high dimensional PDE's}
\author{Carlos Daniel Contreras Quiroz}
\date{Day Month Year}


\begin{document}
\pagenumbering{roman} 

\begin{titlepage}
    \begin{center}
        \vspace*{1cm}
        
        \Huge
        \textbf{A deep learning method for high dimensional PDE's}
        
        \vspace{0.5cm}
        \LARGE
        An application to mean field games\\
        \vspace{0.5cm}
   
        by\\
        \vspace{0.5cm}
    
        \textbf{Carlos Daniel Contreras Quiroz}\\
        \vspace*{1cm}
        Advisor: Mauricio Junca
        
        \vfill
        
        A dissertation submitted in partial fulfillment\\
        of the requirements for the degree of\\
        Master in\\
        Mathematics\\
    
        
        \vspace{1.8cm}

        
        \Large
        at the\\Universidad de los Andes\\
        2023\\
        \vspace{1.0cm}
       
        
    \end{center}
    
\end{titlepage}



\fancyhf{} % clear all header and footer fields
\fancyhead[RO,R]{\thepage} %RO=right odd, RE=right even
\renewcommand{\headrulewidth}{0pt}

\begin{center}
    \Large
    \textbf{Deep learning methods for high dimensional PDE's}
    
    \vspace{0.4cm}
    \large
    An application to N-agent games
    
    \vspace{0.4cm}
    \textbf{Carlos Daniel Contreras Quiroz}
    
    \vspace{0.9cm}
    \textbf{Abstract}
\end{center}
Nada

\chapter*{Acknowledgements}
A mi lulú y mi pancita.Nadita.

\newpage

\tableofcontents
\listoffigures


%\listoffigures


%\listoftables


\chapter{Introduction}
\pagenumbering{arabic} 
Partial differential equations are common tools for modelling phenomena in various disciplines. We can find many of them in physics, biology, engineering and finance. However, their mathematical behavior is as diverse as the areas in which they can be applied. Even if we can establish a partial classification  using some properties of the equations, there cannot exit a complete theory that contains all our understanding about them

By the other hand, in general we have no access to analytical solutions to such models, and thus me may require numerical methods to obtain approximate solutions. To do this, we have well studied tools as finite element, finite differences, finite volumes and spectral methods. There is strong theory that supports rigorously its convergence and when can they be used and when they can fail.
 
However, applying these methods to particular problems is restricted in many cases by the nature of the equations they are applied to. For example, the aforementioned methods suffer from what is called \textit{the curse of dimensionality}, i.e, their computational complexity grows exponentially in the dimension of the problem. Hence, they are restricted to low dimensional cases ($n=1,2,3,4$). This is a problem in areas like financial mathematics or image denoising, where the dimension of the PDE to be solved is determined by the number of underlying assets in the former case, and the size of the image in the latter. Moreover, there may exist other problems besides the dimension of the equation, for example, many non-linearities or complex domains are usually intractable. 

Another area where those problems have appeared is in data analysis and machine learning. For instance, the complexity of some non-linear regression models grows exponentially with the size of the subjacent data. For those problems, we have developed many powerful tools that allow us to deal with high-dimensional and non-linear problems. Particularly, neural networks have been considerably successful to model problems with such complexities \cite{higham_deep_2019}. 

Therefore, trying to replicate this achievement in machine learning in the context of partial differential equations, new perspectives have emerged to approximate solutions using the same set of tools. Between them, we can find Physics Informed Neural Networks (PINNS) \cite{raissi_physics-informed_2019,sirignano_dgm_2018} and Fourier Neural Operators (FNO) \cite{li_fourier_2021}. Empirical evidence suggest that those methods can approximate solutions where classical discretization methods cannot \cite{cuomo_scientific_2022,blechschmidt_three_2021}, despite usually not being competitive in terms of accuracy in the cases where both work. Moreover, there is not yet a well established theoretical framework that justify when and how they should be used, but it is a rapidly evolving subject of study.

In this work we focus on one family of these new methods, named the Deep BSDE methods, that was originally proposed in \cite{han_solving_2018}. They exploit the close relation that exists between forward backward stochastic differential equations and a certain class of non-linear partial differential equations. Hence, our problem can be transformed in a learning one for a neural network approximation that we will train using data associated to a Monte Carlo simulation of diffusion paths.

This document is organized as follows: In Chapter 2 we review some theoretical background that is needed to establish the method, and in Chapter 3 we perform some numerical tests for solving a control problem using the Hamilton-Jacobi-Bellman in high dimensions. Finally, in Chapter 4 we use such tools for solving a game theory problem with interacting agents trying to optimize individual cost functions. 

\chapter{Backward stochastic differential equations and PDEs}
When we deal with deterministic optimal control problems there are two approaches one involving Bellman's dynamic programming principle and the other relying on the Pontryagin's maximum principle. The former approach leads to a partial differential equation named the Hamilton-Jacobi-Bell equation, while the latter leads to a system of ordinary differential equations which are defined backward in time. 


\chapter{Deep Learning Methods for PDEs}
\section{Deep BSDE method}
\subsection{Deep BSDE with one network}

\section{Deep Backward Dynamic Programming}
\section{Deep Splitting}
\section{PINN's and mixed models}

\chapter{Crowd motion modeling}
People run
\section{N-agent games}
\section{Mean Field Games}
\section{Numerical methods}
\subsection{Finite differences}
\subsection{Deep Fictious Play}
\section{An example}

\chapter{Conclusion}
In this work we evidenced the usefulness of various deep learning methods to solve partial differential equations. We applied the to solve optimal control problems using the Hamilton-Jacobi-Bellman equation in dimension 6, which is not accessible through traditional discretization methods. We implemented them in the Python programming language using the Pytorch library for automatic differentiation and neural networks tools.

These methods are very prone to error, as there are many hyperparameters that we need to tune to achieve optimal convergence. When we select them wrong, the process could even diverge. There is not a general formula to make them work in every case.

There is further work we did not accomplish due to time limitations. Our original objective was to solve an N-player game with a bigger N $(N>100)$ to compare with solutions provided by the mean field approach, see for example \cite{achdou_mean_2020}. However, boundary conditions supposed a hard problem we have not solved yet, but we expect to figure out in a near future. Also, we may solve big centralized optimal control problems to compare them with the equivalent in the limit McKean-Vlasov control problem, see \cite{carmona_control_2013}. We may be able to study the degeneration of optimal state when there is not a centralized control, and instead every player is trying to optimize its own function.

We have many ideas to be implemented. Here we list some of them 
\begin{enumerate}
	\item To take into account boundary conditions while being able to access accurately derivatives of the solution to calculate controls, we plan to try reflecting and stopping the process upon contact with a certain part of the boundary, adjusting the Deep BSDE method to include these new conditions. Preliminary results suggest that it may work, but further analysis should be done to justify this procedure theoretically.
	\item The DGM was not useful to solve bigger control problems with interaction between the particles. Our hypothesis is that this difficulty is due to the sampling step of interior points that not captures sufficient information for the solution to be accurate. Our idea is to device a method to sample points in important regions of the $N$-dimensional domain, where the solution is not well approximated by uniform sampling.
	\item In a similar spirit, maybe using a different way to produce sample paths driving them to certain regions of the domain that may be problematical would benefit convergence speed, see for example \cite{nusken_solving_2023}. 
\end{enumerate}    

\begin{appendices}
	\chapter{Neural Networks}
	\input{chapters/appendixNN}
	\chapter{Stochastic Control}
	\label{chp:ApendixStochasticControl}
\newtheorem{thm}{Theorem}[chapter]% theorem counter resets every \subsection
%\renewcommand{\thethm}{\arabic{thm}}
In this appendix we review, without proofs, the basics of stochastic optimal control leading to the Hamilton-Jacobi-Bellman equation used in this work, and give the linear-quadratic regulator as an example of this theory. We follow \cite{pham_continuous-time_2009}.
\section*{The Hamilton-Jacobi-Bellman Equation}
Suppose that we want to control a process $X_t\in \bbR^n$ that satisfies a stochastic differential equation driven by $d$-dimensional Brownian motion of the form 
\begin{equation}
	\begin{split}
		&dX_t=\mu(X_t,\alpha_t)dt+\sigma(X_t,\alpha_t)dW_t\\
		&X_0=x,
	\end{split}
\end{equation} 
with a control function $\alpha_t$ taking values in some admissible space $A$. From now on we assume $\mu$ and $\sigma$ satisfy the standard Lipschitz conditions required for a solution to this equation exist. We want to choose such control so that the total benefit functional given by 
\begin{equation}
	J(\alpha_t)=\mathbb{E}\left[\int_{0}^{T}f(s,X_s,\alpha_s) ds +g(X_T)\right].
\end{equation}
is maximum over all possible control functions. Here, the function $f$ is called running cost and $g$ is called terminal cost. At any time $t$, we can choose the controller using only information observed before $t$, as we are unable to foretell the future due to the system's randomness. Therefore, we require $\alpha_t$ to be $\mathcal{F}_t$-adapted and define the set of feasible controls as $\mathcal{A}([0,T])=\{\alpha:[0,T]\times \Omega\to A\}$.\hlc[Definir tipos de control morkoviano open,...]{arg2}

A stochastic control problem consists on finding $\hat{\alpha}\in \mathcal{A}([0,T])$ such that
\begin{equation}
	J(\hat{\alpha})=\sup_{\alpha_t\in\mathcal{A}([0,T])} J(\alpha).
\end{equation}

We need the following definitions. For all $(t,x)\in \bbR^{+}\times \bbR^n$ and $\alpha \in \mathcal{A}([0,T])$, we denote by $X_s^{t,x,\alpha}$ the solution to the SDE
\begin{equation}
	\begin{split}
	&dX_{s}^{t,x,\alpha}=\mu(X_{s}^{t,x,\alpha},\alpha_s)ds+\sigma(X_{s}^{t,x,\alpha},\alpha_s)dW_s\\
	&X_t^{t,x,\alpha}=x.
	\end{split}
\end{equation}
Now, we define the value functional starting at time $t$ and position $x$ as
\begin{equation}
	J(t,x,\alpha)=\mathbb{E}\left[\int_{t}^{T}f(s,X_s^{t,x,\alpha},\alpha_s) ds +g(X_T^{t,x,\alpha})\right].
\end{equation}
and the \textit{value function} $V(t,x)$ as
\begin{equation}
	V(t,x)=\sup_{\alpha\in\mathcal{A}[t,T]}J(t,x,\alpha),
\end{equation}
which is the expected optimal reward starting the process at time $t$ and point $x$.

To solve the stochastic control problem, we follow the approach based on the \textit{dynamic programming principle}, which states informally that 

\begin{center}
	"An optimal policy has the
	property that whatever the
	initial state and initial decision are, the remaining decisions must constitute an
	optimal policy with regard
	to the state resulting from
	the first decision"
	Richard Bellman
\end{center}
and can be translated in the following theorem
\begin{thm}[Stochastic dynamic programming \cite{pham_continuous-time_2009}]
	For all $0\leq t\leq s\leq T$ and $x\in\bbR^n$ we have that the value function $V(t,x)$ satisfies \hlc[Revisar esto, la s no tiene sentido]{}
	\begin{equation}
		V(t,x)=\sup_{\alpha\in \mathcal{A}[t,s]}\expect*{\int_{t}^{s}f(r,X_r^{t,x,\alpha},\alpha_r)dr+V(s,V_s^{t,x,\alpha})},
	\end{equation}
from which a infinitesimal version can be derived, named the Hamilton-Jacobi-Bellman equation 
\begin{equation}
	\begin{split}
		&\dpartial{V}{t}+\sup_{a\in A}\{\mathcal{L}^a[V](t,x)+f(t,x,a)\}=0\\
		&V(T,x)=g(x),
	\end{split}
\end{equation}
where $\mathcal{L}$ is the infinitesimal generator of the controlled process $X_t$ given by
\begin{equation}
	\mathcal{L}^a[V](t,x)=\mu(x,a)\cdot D_x V(t,x)+\frac{1}{2}\Tr(\sigma(x,a)\sigma(x,a)'D_{xx}V(t,x)).
\end{equation}
\end{thm}

We can also write the Hamilton-Jacobi-Bellman equation as 
\begin{equation}
		\begin{split}
		&\dpartial{V}{t}+H(t,x,D_x V,D_{xx} V)=0\\
		&V(T,x)=g(x),
	\end{split}
\end{equation}
where the function $H(t,x,p,M)$ is the \textit{hamiltonian} defined as
\begin{equation}
	H(t,x,p,M)=\sup_{a\in A}\{\mu(x,a)\cdot p+\frac{1}{2}\Tr (\sigma\sigma'(x,a)M)+f(t,x,a)\}.
\end{equation}

Note that we assume implicitly that the supremums appearing in these equations exists, but this condition is not necessary as pointed in \cite{pham_continuous-time_2009}.

Solving this equation the Hamilton-Jacobi-Bellman equation for the function $V(t,x)$ can be used to construct optimal controls for the original problem as will be shown below with the linear-quadratic regulator. However, we need a result stating that a solution to such equation is in fact the desired value function
\begin{thm}[Verfication theorem \cite{pham_continuous-time_2009}]
	Let $w$ be a function in $C^{1,2}\left([0, T) \times \mathbb{R}^n\right) \cap C^0\left([0, T] \times \mathbb{R}^n\right)$, and satisfying a quadratic growth condition, i.e. there exists a constant $C$ such that
	$$
	|w(t, x)| \leq C\left(1+|x|^2\right), \quad \forall(t, x) \in[0, T] \times \mathbb{R}^n
	$$
	(i) Suppose that
	$$
	\begin{aligned}
		-\frac{\partial w}{\partial t}(t, x)-\sup _{a \in A}\left[\mathcal{L}^a w(t, x)+f(t, x, a)\right] & \geq 0, \quad(t, x) \in[0, T) \times \mathbb{R}^n, \\
		w(T, x) & \geq g(x), \quad x \in R^n .
	\end{aligned}
	$$
	Then $w \geq v$ on $[0, T] \times \mathbb{R}^n$.
	
	(ii) Suppose further that $w(T,x)=g(x)$ , and there exists a measurable function $\hat{\alpha}(t, x)$, $(t, x) \in[0, T) \times \mathbb{R}^n$, valued in $A$ such that
	$$
	\begin{aligned}
		-\frac{\partial w}{\partial t}(t, x)-\sup _{a \in A}\left[\mathcal{L}^a w(t, x)+f(t, x, a)\right] & =-\frac{\partial w}{\partial t}(t, x)-\mathcal{L}^{\hat{\alpha}(t, x)} w(t, x)-f(t, x, \hat{\alpha}(t, x)) \\
		& =0
	\end{aligned}
	$$
	the $S D E$
	$$
	d X_s=b\left(X_s, \hat{\alpha}\left(s, X_s\right)\right) d s+\sigma\left(X_s, \hat{\alpha}\left(s, X_s\right)\right) d W_s
	$$
	admits a unique solution, denoted by $\hat{X}_s^{t, x}$, given an initial condition $X_t=x$, and the process $\left\{\hat{\alpha}\left(s, \hat{X}_s^{t, x}\right) t \leq s \leq T\right\}$ lies in $\mathcal{A}(t, x)$. Then
	$$
	w=v \quad \text { on }[0, T] \times \mathbb{R}^n,
	$$
	and $\hat{\alpha}$ is an optimal Markovian control.
\end{thm}
\section*{The linear-quadratic regulator (LQR)}
\end{appendices}
%\appendix



\printbibliography
\end{document}
