Las ecuaciones en derivadas parciales aparecen comúnmente como herramientas útiles para la modelación en múltiples disciplinas. Se encuentran frecuentemente aplicaciones en ciencias naturales como la física y biología, en diseño en ingeniería , y también en áreas como la economía y finanzas. Sin embargo, las propiedades matemáticas de las ecuaciones que aparecen son tan diversas como las áreas en que se aplican, y aunque se pueden clasificar parcialmente según algunas de sus características, no podría existir una teoría completa que describa nuestro conocimiento sobre estas.\\

Por otro lado, las soluciones analíticas a estos modelos generalmente no están a nuestro alcance, por lo que es necesario recurrir a métodos numéricos para obtener aproximaciones. Para esto, usualmente se recurre a métodos clásicos como diferencias finitas, elementos finitos, volúmenes finitos o métodos espectrales, para los cuales existe una amplia teoría que soporta y justifica rigurosamente su funcionamiento.\\

No obstante, la aplicación de estos métodos a problemas particulares a veces se restringe por propiedades especificas de la ecuación que se resuelve. Por ejemplo, los métodos mencionados sufren de la maldición de la dimensionalidad (\textit{"the curse of dimentionality"}), esto es, su complejidad computacional escala exponencialmente en la dimensión del problema, por lo que su uso se restringe a problemas de dimensión baja ($n=1,2,3,4$). Lo anterior dificulta su implementación en aplicaciones como valoración en matemática financiera, donde la dimensión del problema está determinada por el número de activos considerados . También, su eficiencia computacional se reduce considerablemente conforme se aumenta la complejidad de los dominios en que se resuelven, o por las no-linealidades que aparecen, como es el caso de la ecuación de Navier-Stokes modelando flujos turbulentos.\\

Otra área en donde estos inconvenientes aparecen es en el análisis de datos y aprendizaje de maquinas. Por ejemplo, la complejidad de algunos modelos de regresión no lineal crece exponencialmente con el tamaño de los datos subyacentes. Para este tipo de problemas se han desarrollado herramientas poderosas que permiten modelar problemas en altas dimensiones y con posibles no linealidades. Entre estas, las redes neuronales han demostrado ser de gran utilidad como modelo para representar funciones con estas complejidades\cite{higham_deep_2019}.\\

En consecuencia, intentando replicar el éxito obtenido con estas herramientas en aprendizaje de máquinas, recientemente han surgido nuevas perspectivas para aproximar soluciones de ecuaciones en derivadas parciales usando estas mismas herramientas. Entre estas se encuentran las PINNs (Physics Informed Neural Networks)\cite{PINNs,PINNS2}, FNO (Fourier Neural Operators)\cite{li_fourier_2021}, y DGM (Deep Garlekin Method)\cite{sirignano_dgm_2018}. La evidencia práctica muestra que estos métodos pueden proporcionar soluciones en casos donde los clásicos no \cite{cuomo_scientific_2022,blechschmidt_three_2021}, a pesar de usualmente no competir con su eficiencia en las situaciones donde los últimos sí aplican. Además, se ha venido desarrollando un marco teórico riguroso que permite justificar su aplicación en situaciones específicas.  \\