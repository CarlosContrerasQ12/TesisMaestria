% !TeX spellcheck = es_ES
\documentclass{article}
\usepackage[T1]{fontenc}
\usepackage[utf8]{inputenc}
\usepackage[margin=1in]{geometry}
\usepackage[most]{tcolorbox}
\usepackage{amsmath}
\usepackage{amssymb}
\usepackage{parskip}

\setlength{\parindent}{0cm}

\usepackage{xcolor,pifont}
\newcommand*\colourcheck[1]{%
	\expandafter\newcommand\csname #1check\endcsname{\textcolor{#1}{\ding{52}}}%
}

\newcommand*\colourwrong[1]{%
	\expandafter\newcommand\csname #1wrong\endcsname{\textcolor{#1}{\ding{55}}}%
}

\newcommand{\dpartial}[2]{\frac{\partial #1}{\partial #2}}

\colourcheck{blue}
\colourcheck{green}
\colourcheck{red}

\colourwrong{red}

\newcommand{\HRule}{\rule{\linewidth}{0.5mm}}
\newcommand{\Hrule}{\rule{\linewidth}{0.3mm}}

\makeatletter% since there's an at-sign (@) in the command name
\renewcommand{\@maketitle}{%
	\parindent=0pt% don't indent paragraphs in the title block
	\centering
	{\Large \bfseries\textsc{\@title}}
	\HRule\par%
	\textit{\@author \hfill \@date}
	\par
}
\makeatother% resets the meaning of the at-sign (@)

\usepackage[sorting=none]{biblatex}
\addbibresource{PDEsML.bib}

\usepackage{listings}

\definecolor{codegreen}{rgb}{0,0.6,0}
\definecolor{codegray}{rgb}{0.5,0.5,0.5}
\definecolor{codepurple}{rgb}{0.58,0,0.82}
\definecolor{backcolour}{rgb}{0.95,0.95,0.92}

\lstdefinestyle{mystyle}{
	backgroundcolor=\color{backcolour},   
	commentstyle=\color{codegreen},
	keywordstyle=\color{magenta},
	numberstyle=\tiny\color{codegray},
	stringstyle=\color{codepurple},
	basicstyle=\ttfamily\footnotesize,
	breakatwhitespace=false,         
	breaklines=true,                 
	captionpos=b,                    
	keepspaces=true,                 
	numbers=left,                    
	numbersep=5pt,                  
	showspaces=false,                
	showstringspaces=false,
	showtabs=false,                  
	tabsize=2
}

\lstset{style=mystyle}

\newenvironment{recordar}[1][Recordar]
{\begin{tcolorbox}[breakable,colback=red!10!white,colframe=red!50!blue,title=Recordar: #1 ,enhanced jigsaw]
	
}{\end{tcolorbox}}

\title{Notas}
\author{Carlos Daniel Contreras Quiroz}
\begin{document}
	\maketitle% prints the title block
	\vspace{5mm}
	\section*{To do list}
	\begin{enumerate}
		\item Entender BSDE 
		\item Escribir conexión PDE-BSDE
		\item Programar reflexión en frontera
		\item Deep Fictious Play
		\item Escribir Crowd motion
		\item Simular diferencias
		\item Escribir simulaciones
		\item Escribir apendice Neural Networks
	\end{enumerate}
    \HRule\par
	\section*{Problemas}
	\begin{enumerate}
		\item El repositorio del review corre muy lento \bluecheck
		\item Las condiciones de frontera no son iguales
		\begin{itemize}
			\item Reflejar los caminos puede funcionar \redwrong
		\end{itemize}
		\item Parece que DeepBSDE no funciona con dimensiones de más de 100
		\begin{itemize}
			\item Podría usarse el deep Backward de Pham \cite{hure_deep_2020}, se puede entrenar basado en modelos anteriores. No hay código
			\item O tambien deep splitting de Beck \cite{beck_deep_2021}. Si hay código
		\end{itemize}
	\end{enumerate}
    \HRule\par
    \section*{Ideas}
    \begin{itemize}
    	\item Hacerlo en flux
    	\item Hacerlo en equinox/jax
    	\item Buscar un modelo de Cucker Smale abierto
    	\item Buscar un método para resolver problemas en horizonte infinito (actor-critic podría funcionar)
    	\item Resolver MFG con deep learning
    \end{itemize}
    \HRule\par
    \section*{Preguntas}
	\begin{enumerate}
	\item ¿En que sentido converge a la solución?
	\item ¿Qué pasaría si se cambia la discretización de Euler por una mejor?¿Afecta el modo de convergencia?
	\end{enumerate}
    \HRule\par
    \section*{Probabilidad}
    \subsection*{Preguntas}
    \begin{itemize}
    	\item Es $E[\int_{0}^{T} |f|^2]<\infty$ equivalente a que $\int_{0}^{T}|f|^2<\infty$ a.s?
    	\item Revisar desigualdad de doob en prueba
    	\item Typo al aplicar Bukholder
    \end{itemize}
    \subsection*{Para recordar}
    \begin{recordar}[Filtración aumentada]
    	A una filtración $\mathbb{F}=(\mathcal{F}_t)_t$ en un espacio de probabilidad $(\Omega,\mathcal{A},\mathbb{P})$ le corresponde una filtración continua a la derecha $\mathbb{F}^+=(\mathcal{F}_t)_t=\cap_{t<s}(\mathcal{F}_s)$. $\mathbb{F}$ se dice continua a la derecha si $\mathbb{F}^+=\mathbb{F}$.
    	
    	Sea $\mathcal{N}_P=\{A\subseteq \omega | A\subseteq B \text{ para algún } B\in \mathbb{F} \text{ con } P(B)=0\}$. $\mathbb{F}$ se dice completa si $\mathcal{F}_t$ contiene $\mathcal{N}_P$ para todo $t$.
    	
    	Una filtración continua a la derecha y completa se llama filtración aumentada.
    \end{recordar}
	\section*{Deep BSDE}
	Vamos a intentar resolver ecuaciones del tipo 
	\begin{equation}
		\begin{gathered}
			\frac{\partial u}{\partial t}(t, x)+\frac{1}{2} \operatorname{Tr}\left(\sigma \sigma^{\mathrm{T}}(t, x)\left(\operatorname{Hess}_x u\right)(t, x)\right)+\nabla u(t, x) \cdot \mu(t, x) \\
			+f\left(t, x, u(t, x), \sigma^{\mathrm{T}}(t, x) \nabla u(t, x)\right)=0
		\end{gathered}
	\end{equation}
con la condición final $u(T,x)=g(x)$.

Vamos a realizar una aproximación con la fórmula de Feynman-Kac. Esto es, la solución de la ecuación anterior viene dada por 

\begin{equation}
	\begin{aligned}
		& u\left(t, X_t\right)-u\left(0, X_0\right) \\
		=&-\int_0^t f\left(s, X_s, u\left(s, X_s\right), \sigma^{\mathrm{T}}\left(s, X_s\right) \nabla u\left(s, X_s\right)\right) d s \\
		&+\int_0^t\left[\nabla u\left(s, X_s\right)\right]^{\mathrm{T}} \sigma\left(s, X_s\right) d W_s .
	\end{aligned}
\end{equation}
donde $X$ resuelve la ecuación diferencial estocástica
\begin{equation}
	X_t=\xi+\int_0^t \mu\left(s, X_s\right) d s+\int_0^t \sigma\left(s, X_s\right) d W_s
\end{equation}
Vamos a estudiar primero el caso de HJB del repositorio. Acá intentamos controlar el proceso 100-dimensional
\begin{equation}
	dX_t=2\sqrt{\lambda}m_t dt +\sqrt{2}dt \quad X(0)=x \quad t\in (0,T)
\end{equation}
a través del control $m_t$, con el funcional de costo
\begin{equation}
	J(m_t)=\mathbb{E}\left[\int_{0}^{T}||m_t||^2 dt +g(X_T)\right].
\end{equation}
\begin{recordar}[Hamilton-Jacobi-Bellman]
	Para un proceso controlado
	\begin{equation*}
		\begin{split}
			&dX_x=\mu(X,u)dt+\sigma(X,u)dW\\
			&X(0)=x
		\end{split}
	\end{equation*}
con función de costo
\begin{equation*}
	J(t,x,u)=\mathbb{E}\left[\int_{t}^{T}f(X_x,u) dt +g(X_x)\right].
\end{equation*}
la ecuación de HJB para la función de valor
\begin{equation*}
	V(t,x)=\sup_{u\in \mathcal{A}}J(t,x,u)
\end{equation*}
es
\begin{equation*}
	\dpartial{V}{t}+\sup_{u\in A}\{ \mathcal{L}^{u}(V)+f(x,u)\}=0
\end{equation*}
donde 
\begin{equation*}
	\mathcal{L}^{u}(V)(t,x)=\mu(x,u)\cdot \nabla V(t,x) +\frac{1}{2}Tr(\sigma(t,x)\sigma^{T}(t,x)\nabla^2 V(t,x))
\end{equation*}
con la condición final
\begin{equation*}
	u(T,x)=g(x)
\end{equation*}
Esto también puede escribirse con el Hamiltoniano
	\begin{equation*}
		H(t,x,p,M)=\sup{u\in \mathcal{A}}\{\mu(x,u)\cdot p+\frac{1}{2}Tr[\sigma(t,x)\sigma^{T}(t,x)M]+f(x,u)\}
	\end{equation*}
y asumiendo que el control no se aplica a la volatilidad se escribe
\begin{equation*}
	\dpartial{u}{t}+H(t,x,\nabla u)+\frac{1}{2}Tr[\sigma(t,x)\sigma^{T}(t,x)\nabla^2 u]=0
\end{equation*}

\end{recordar}


\begin{lstlisting}[language=Python]
	import numpy as np
	
	def incmatrix(genl1,genl2):
	m = len(genl1)
	n = len(genl2)
	M = None #to become the incidence matrix
	VT = np.zeros((n*m,1), int)  #dummy variable
	
	#compute the bitwise xor matrix
	M1 = bitxormatrix(genl1)
	M2 = np.triu(bitxormatrix(genl2),1) 
	
	for i in range(m-1):
	for j in range(i+1, m):
	[r,c] = np.where(M2 == M1[i,j])
	for k in range(len(r)):
	VT[(i)*n + r[k]] = 1;
	VT[(i)*n + c[k]] = 1;
	VT[(j)*n + r[k]] = 1;
	VT[(j)*n + c[k]] = 1;
	
	if M is None:
	M = np.copy(VT)
	else:
	M = np.concatenate((M, VT), 1)
	
	VT = np.zeros((n*m,1), int)
	
	return M
\end{lstlisting}
\printbibliography
\end{document}